\section{Metriche di Valutazione}

Per valutare l'efficacia di un classificatore è necessario quantificare quanto esso sia capace di effettuare previsioni corrette\
su dati diversi da quelli con cui è stato appreso.\
Uno strumento utile a questo scopo è la \textbf{matrice di confusione}\mycite{MurelKavlakoglu_2025}.\
\
Nella matrice, le righe rappresentano i valori effettivi mentre le colonne indicano quelli predetti;\
in letteratura è possibile incontrare anche la rappresentazione inversa.\
La matrice ha tante righe e colonne quante sono le classi presenti nel dataset.\
L'intersezione tra la riga $i$ e la colonna $j$ fornisce il numero di istanze della classe\
$i$ che sono state classificate come classe $j$. In particolare, gli elementi della diagonale\
rappresentano le istanze correttamente predette.\
\
A partire dalla matrice di confusione è possibile calcolare diverse metriche di valutazione,\
tra cui la \textbf{precision}, la \textbf{recall} e l'\textbf{F1-score} per ciascuna classe.\
È inoltre possibile derivare le corrispondenti misure aggregate a livello \textit{macro} e\
\textit{micro}, utili per valutazioni globali sul dataset.\
Prima di entrare nel dettaglio delle metriche tornano utili i concetti di \textbf{true positive},\
\textbf{true negative}, \textbf{false positive} e \textbf{false negative} considerando la matrice di confusione $M$ avente $n$ classi.

I \textbf{true positive} di una classe $k$ equivalgono al numero di istanze di classe $k$ che
sono state predette come classe $k$:
\[
    TP_{k} = M_{k,k}
\]

I \textbf{false positive} di una classe $k$ rappresentano il numero di istanze appartenenti a
classi diverse da $k$ che sono state erroneamente predette come $k$:
\[
    FP_{k} = \sum_{\substack{i=1 \\ i \neq k}}^{n} M_{i,k}
\]

I \textbf{false negative} di una classe $k$ rappresentano il numero di istanze appartenenti a
$k$ che sono state erroneamente classificate come un'altra classe:
\[
    FN_{k} = \sum_{\substack{j=1 \\ j \neq k}}^{n} M_{k,j}
\]

I \textbf{true negative} di una classe $k$ rappresentano il numero di istanze appartenenti a
classi diverse da $k$ che sono state correttamente classificate come non $k$:
\[
    TN_{k} = \sum_{\substack{i=1 \\ i \neq k}}^{n}
    \sum_{\substack{j=1 \\ j \neq k}}^{n} M_{i,j}
\]

La \textbf{precision} di una classe $k$ è il rapporto tra i veri positivi e la
somma dei veri positivi e dei falsi positivi:
\[
    \mathrm{Precision}_k = \frac{TP_k}{TP_k + FP_k}
\]

La \textbf{recall} (o sensibilità) di una classe $k$ è il rapporto tra i veri
positivi e la somma dei veri positivi e dei falsi negativi:
\[
    \mathrm{Recall}_k = \frac{TP_k}{TP_k + FN_k}
\]

L'\textbf{F1-score} di una classe $k$ è la media armonica tra precision e recall:
\[
    \mathrm{F1}_k = 2 \cdot \frac{\mathrm{Precision}_k \cdot \mathrm{Recall}_k}
    {\mathrm{Precision}_k + \mathrm{Recall}_k}
\]

Le metriche \textbf{Precision}, \textbf{Recall} e \textbf{F1-Score} calcolate per ciascuna classe
possono essere aggregate per ottenere valori complessivi sul modello intero. Due approcci
comuni di aggregazione sono la media \textit{macro} e la media \textit{micro}:

\[
    \mathrm{Macro_{precision}} = \frac{\sum_{i}^{n} Precision_{i}}{n}
\]

\[
    \mathrm{Macro_{recall}} = \frac{\sum_{i}^{n} Recall{i}}{n}
\]

\[
    \mathrm{Macro_{F1}} = \frac{\sum_{i}^{n} F1{i}}{n}
\]

\[
    \mathrm{Micro_{Precision}} = \frac{\sum_{i}^{n} TP_{i}}{\sum_{i}^{n} TP_{i} + \sum_{i}^{n} FP_{i}}
\]

\[
    \mathrm{Micro_{Recall}} = \frac{\sum_{i}^{n} TP_{i}}{\sum_{i}^{n} TP_{i} + \sum_{i}^{n} FN_{i}}
\]

\[
    \begin{aligned}
        TP               & = \sum_{i=1}^{n} TP_i,                                                                  & FP            & = \sum_{i=1}^{n} FP_i, & FN & = \sum_{i=1}^{n} FN_i, \\[1mm]
        \text{Precision} & = \frac{TP}{TP + FP},                                                                   & \text{Recall} & = \frac{TP}{TP + FN},                                \\[1mm]
        \text{Micro-F1}  & = 2 \cdot \frac{\text{Precision} \cdot \text{Recall}}{\text{Precision} + \text{Recall}}
    \end{aligned}
\]

Una metrica che riassume la prestazione globale è la \textit{overall accuracy}:

\[
    \mathrm{OverallAccuracy} = \frac{\sum_{i}^{n} M_{i,i}}{ \sum_{\substack{i=1}}^{n}
        \sum_{\substack{j=1 }}^{n} M_{i,j}}
\]
