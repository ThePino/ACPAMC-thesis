\newcommand{\mtrdescription}[1]{Matrici di confusione sul dataset \textbf{#1}}
\newcommand{\clscaption}[2]{\textit{Precision}, \textit{Recall} e \textit{F1-score} sul dataset {\textbf{#1}} con modello {\textit{#2}}}
\newcommand{\grpdescription}[1]{Overall accuracy, micro e macro metriche sul dataset \textbf{#1}}
\newcommand{\classmetrics}[3]{In \autoref{#1} e in \autoref{#2} le metriche \textit{Precision}, \textit{Recall} e \textit{F1-score} per \textit{XGBoost} e \textit{RandomForest} per il dataset \textbf{#3}.}
\newcommand{\globalmetrics}[2]{In \autoref{#1} le metriche \textit{Overall accuracy},  \textit{Micro precision}, \textit{Micro recall}, \textit{Micro F1-score}, \textit{Macro precision}, \textit{Macro recall} e \textit{Macro F1-score} per il dataset \textbf{#2}.}

\section{Risultati}

Per analizzare in dettaglio il comportamento dei modelli sulle diverse classi, sono state calcolate le matrici di confusione.\
Esse permettono di visualizzare le corrette classificazioni (diagonale principale) e gli errori di classificazione.
%%%%%%%%%%%%%%%%%%%%%%%%%%%%%%%%%%% APIMDS $$$$$$$$$$$$$$$$$$$$$$$$$$$$$$$$$$$$$$$$$$$$$$$$$
Le matrici di confusione ottenute per il dataset \textbf{apimds} sono illustrate in \autoref{fig:apimds-mtrx}.

\begin{figure}[ht]
    \centering
    \adjustbox{max width=1.4\linewidth, max height=1\textheight}{%
        \includesvg[width=\textwidth]{validazione-empirica/imgs/apimds-confusion-matrix.svg}
    }
    \caption{\mtrdescription{apimds}}
    \label{fig:apimds-mtrx}
\end{figure}

Le classi meglio identificate risultano \textit{Trojan}, \textit{Malware} e \textit{Virus} in entrambi i modelli.\
Queste classi coincidono con quelle più rappresentate nel dataset, suggerendo che la distribuzione dei dati influisce in modo significativo\
sulla capacità di apprendimento del modello.\
Si osservano buone prestazioni anche sulle classi minoritarie,\
il che lascia supporre che tali campioni presentino pattern di chiamate API più distintivi rispetto alle altre categorie.\
Per quanto riguarda le classi diverse da \textit{Trojan}, il modello quando predice erroneamente, tende a classificarle come \textit{Trojan}.\
Nel complesso, i due classificatori mostrano un comportamento simile, ma \textit{RandomForest} ottiene risultati leggermente migliori.\
Non emergono pattern di errore anomali: gli errori si distribuiscono in modo coerente con la rappresentatività delle classi.


\begin{figure}[ht]
    \centering
    \adjustbox{max width=\linewidth, max height=0.8\textheight, keepaspectratio}{%
        \includesvg{validazione-empirica/imgs/apimds_XGBoost_metrics.svg}
    }
    \caption{\clscaption{apimds}{XGBoost}}
    \label{fig:apimds-cls-xgb}
\end{figure}


\begin{figure}[ht]
    \centering
    \includesvg[width=\linewidth, keepaspectratio]{validazione-empirica/imgs/apimds_RandomForest_metrics.svg}
    \caption{\clscaption{apimds}{RandomForest}}
    \label{fig:apimds-cls-rf}
\end{figure}

\FloatBarrier

In \autoref{fig:apimds-cls-xgb} vengono illsturate le metriche \textit{precision}, \textit{recall} e \textit{f1-score} per il modello \textit{XGBoost}.\
Per quanto riguarda la \textit{precision}, il modello presenta valori generalmente buoni, con la maggior parte delle classi che supera il punteggio di $0.73$.\
Relativamente alla \textit{recall}, si notano valori più variabili: il modello eccelle nella classificazione di \textit{Trojan}, ottenendo un punteggio di $0.94$.\
Si evidenziano, tuttavia, valori di recall bassi per le classi \textit{Backdoor}, \textit{Downloader} e \textit{Worm}, che registrano valori pari a $0.35$, $0.56$ e $0.56$.\
Questi risultati si rifletto sull'\textit{F1-score}, che risulta penalizzato per queste classi (ad esempio, $0.48$ per \textit{Backdoor}).\
Parallelamente, in \autoref{fig:apimds-cls-rf} sono mostrati i risultati per il modello \textit{RandomForest}.\
Questo modello presenta una \textit{precision} molto robusta, spesso superiore a \textit{XGBoost}, con la maggior parte delle classi che supera il punteggio di $0.80$ (ad esempio, \textit{Malware} $0.89$ e \textit{Trojan} $0.87$).\
Per quanto concerne la \textit{recall}, si osserva un comportamento analogo a \textit{XGBoost}, con prestazioni eccellenti su \textit{Trojan} ($0.93$) e \textit{Virus} ($0.85$).\
Il modello \textit{RandomForest} cosi come \textit{XGBoost} è penalizzato sulla classe \textit{Backdoor} ($0.40$) ma si nota un leggero miglioramento.\
Le classi \textit{Downloader} e \textit{Worm} ottengono entrambe $0.62$, un risultato superiore al modello precedente.\
Dal confronto diretto dei due modelli emerge che \textit{RandomForest} offre prestazioni complessivamente superiori e più bilanciate su questo dataset.\
Analizzando l'\textit{F1-score}, \textit{RandomForest} ottiene un punteggio uguale o superiore a \textit{XGBoost} in ogni singola classe.\
I vantaggi più evidenti si registrano nelle classi \textit{Virus} (\textit{f1-score} di $0.86$ contro $0.81$) e \textit{Packed} ($0.80$ contro $0.75$).\
Entrambi i modelli eccellono nell'identificazione dei \textit{Trojan} (\textit{F1-score} $0.88$ per entrambi),\
ma condividono una marcata difficoltà nel classificare la classe \textit{Backdoor}.


\begin{figure}[]
    \centering
    \adjustbox{max width=1.4\linewidth, max height=0.8\textheight}{%
        \includesvg[width=\textwidth]{validazione-empirica/imgs/apimds-global-metrics.svg}
    }
    \caption{\grpdescription{apimds}}
    \label{fig:apimds-global}
\end{figure}

\FloatBarrier

In \autoref{fig:apimds-global} sono riportate le metriche \textit{overall accurancy}, \textit{micro precision}, \textit{micro recall}, \textit{micro f1-score},\
\textit{micro precision}, \textit{micro recall}, \textit{micro f1-score} di entrambi i classificatori.\
L'analisi  di queste metriche sul dataset rileva una chiara superiorità del classificatore \textit{RandomForest} rispetto\
a \textit{XGBoost}.
\textit{RandomForest} ottiene una \textit{over accuraccy} di $0.83$ contro il $0.81$ di \textit{XGBoost}.\
Tuttavia, la performance distintiva emerge dal confronto della metrica \textit{Macro F1-score}, che assegna lo stesso peso a tutte le classi,\
comprese quelle minoritarie; \textit{RandomForest} ottiene uno score di $0.74$ rispetto al $0.71$ di \textit{XGBoost}.\
Si evince un forte divario tra le metriche di micro e di macro: sulle metriche di micro si hanno ottimi risultati\
evidenziando una particolare capacità nel predire le classi maggioritarie mentre in quelle macro si notano difficoltà\
su quelle minoritarie. Se ne evince che il dataset è sbilanciato.

%%%%%%%%%%%%%%%%% octack %%%%%%%%%%%%%%%%%%%
Le matrici di confusione ottenute per il dataset \textbf{octack} sono illustrate in \autoref{fig:octack-mtrx}.

\begin{figure}[t]
    \centering
    \adjustbox{max width=1.4\linewidth, max height=1\textheight}{%
        \includesvg[width=\textwidth]{validazione-empirica/imgs/octak-confusion-matrix.svg}
    }
    \caption{\mtrdescription{octack}}
    \label{fig:octack-mtrx}
\end{figure}

\FloatBarrier

Le matrici di confusione mostrano che le classi più riconsociute per il modello \texit{XGBoost} sono \textit{Virus},\
\textit{Downloader} e \textit{}
Il classificatore \textit{RandomForest} ottiene ottimi risultati anche per le classe \textit{Spyware}.
Tuttavia, anche in questo caso, come nel dataset precedente, si osserva una tendenza comune a classificare erroneamente diversi esempi come \textit{Trojan},\
evidenziando ancora una nuova la difficoltà individuare un comportamento comune tra i vari casi di \textit{Trojan}.\
La classe \textit{Adware} risulta invece ben distinta dalla altre, con pochissimi falsi positivi e\
falsi negativi, indice di chiamate API non presenti in altre classi.\
Entrambi i modelli mostrano un comportamento simile, ma \textit{RandomForest} ottiene un numero maggiore di veri positivi complessivi,
segno di una migliore capacità di generalizzazione.
\classmetrics{fig:octack-cls-xgb}{fig:octack-cls-rf}{octack}
%%%%%%%%%%%%% Inizio metriche octack %%%%%%%%%%%%%%%
%%%%%%%%%%%%% Random Forest %%%%%%%%%%%%%%%%%%%%%%%%

\begin{figure}[ht]
    \centering
    \adjustbox{max width=\linewidth, max height=0.8\textheight, keepaspectratio}{%
        \includesvg{validazione-empirica/imgs/octak_XGBoost_metrics.svg}
    }
    \caption{\clscaption{octack}{XGBoost}}
    \label{fig:octack-cls-xgb}
\end{figure}


\begin{figure}[ht]
    \centering
    \includesvg[width=\linewidth, keepaspectratio]{validazione-empirica/imgs/octak_RandomForest_metrics.svg}
    \caption{\clscaption{octack}{RandomForest}}
    \label{fig:octack-cls-rf}
\end{figure}

\FloatBarrier

%%%%%%%%%%%%%%% FINE METRICHE CLASSE octack %%%%%%%%%%%%%%%%%%%%%%%%%%%%%%

Il modello \textit{RandomForest} ottiene prestazioni migliori in ogni metrica tranne che per la \textit{precision} sulla classe \textit{Downloader}.
Per entrambi i modelli le classi \textit{Trojan} e \textit{Spyware} risultano le più difficoltose da predire.
Al contrario la classe \textit{Adware} è quella meglio riconosciuta da entrambi i modelli, con metriche elevate e bilanciate.
Nel complesso \textit{RandomForest} tende a generalizzar meglio rispetto a \textit{XGBoost}.
\globalmetrics{fig:octack-global}{octack}

\begin{figure}[t]
    \centering
    \adjustbox{max width=1.4\linewidth, max height=1\textheight}{%
        \includesvg[width=\textwidth]{validazione-empirica/imgs/octak-global-metrics.svg}
    }
    \caption{\grpdescription{octack}}
    \label{fig:octack-global}
\end{figure}

\FloatBarrier

Le metriche globali confermano \textit{RandomForest} come il classificatore più ottimale.

%%%%%%%%%%%%%%%%% MPASCO %%%%%%%%%%%%%%%%%%%

Le matrici di confusione ottenute per il dataset \textbf{mpasco} sono illustrate in \autoref{fig:mpasco-mtrx}.

\begin{figure}[t]
    \centering
    \adjustbox{max width=1.4\linewidth, max height=1\textheight}{%
        \includesvg[width=\textwidth]{validazione-empirica/imgs/mpasco-confusion-matrix.svg}
    }
    \caption{\mtrdescription{mpasco}}
    \label{fig:mpasco-mtrx}
\end{figure}

\FloatBarrier

Il dataset \textbf{mpasco} presenta solo due classi, \textit{Goodware} e \textit{Malware}, ricadendo nel compito di classificazione binaria.\
I falsi negativi (\textit{Malware} classificati come \textit{Goodware}) risultano più numerosi rispetto ai falsi positivi (\textit{Goodware} classificati come \textit{Malware}).\
In un contesto di sicurezza informatica, tuttavia, sarebbe preferibile un comportamento che penalizzi maggiormente i falsi negativi:\
un \textit{Malware} non rilevato ha conseguenze più gravi rispetto ad un \textit{Goodware} non rilevato.\
Entrambi i modelli hanno ottime capacità di distinzione tra le due classi.

%%%%%%%%%%%%% Inizio metriche mpasco %%%%%%%%%%%%%%%
\begin{figure}[ht]
    \centering
    \adjustbox{max width=\linewidth, max height=0.8\textheight, keepaspectratio}{%
        \includesvg{validazione-empirica/imgs/mpasco_XGBoost_metrics.svg}
    }
    \caption{\clscaption{mpasco}{XGBoost}}
    \label{fig:mpasco-cls-xgb}
\end{figure}


\begin{figure}[ht]
    \centering
    \includesvg[width=\linewidth, keepaspectratio]{validazione-empirica/imgs/mpasco_RandomForest_metrics.svg}
    \caption{\clscaption{mpasco}{RandomForest}}
    \label{fig:mpasco-cls-rf}
\end{figure}

\FloatBarrier

%%%%%%%%%%%%%%% FINE METRICHE CLASSE mpasco %%%%%%%%%%%%%%%%%%%%%%%%%%%%%%

Le metriche per classe confermano l'ottima capacità dei modelli nel distinguere \textit{Goodware} e \textit{Malware} ottenendo valori
di metriche pari o uguali a $0.94$.

\begin{figure}[t]
    \centering
    \adjustbox{max width=1.4\linewidth, max height=1.2\textheight}{%
        \includesvg[width=\textwidth]{validazione-empirica/imgs/mpasco-global-metrics.svg}
    }
    \caption{\grpdescription{mpasco}}
    \label{fig:octack-mtrx-rf}
\end{figure}

\FloatBarrier

Le metriche globali rispecchiano le prestazioni ottimali di quelle di classe.


%%%%%%%%%%%%%%%%% QUOVADIS %%%%%%%%%%%%%%%%%%%
Le matrici di confusione ottenute per il dataset \textbf{quovadis} sono le seguenti:

\begin{figure}[t]
    \centering
    \adjustbox{max width=1.4\linewidth, max height=1\textheight}{%
        \includesvg[width=\textwidth]{validazione-empirica/imgs/quovadis-confusion-matrix.svg}
    }
    \caption{\mtrdescription{quovadis}}
    \label{fig:octack-mtrx-rf}
\end{figure}

\FloatBarrier

Il modello \textit{XGBoost} mostra una leggera tendenza ad assegnare con maggiore facilità la classe \textit{Malware},\
indicando un piccolo bias verso questa etichetta.\
In linea generale, tale comportamento non sarebbe necessariamente negativo, poiché in un contesto di sicurezza è preferibile classificare erroneamente\
un file legittimo come sospetto piuttosto che il contrario.\
Tuttavia, in questo caso, tale bias non porta benefici: \textit{XGBoost} tende infatti a confondere un numero maggiore di \textit{Malware} e \textit{Backdoor} come \textit{Goodware} rispetto a \textit{RandomForest}.\
Questo implica che il modello finisce per sottostimare la presenza di file malevoli,\
aumentando i falsi negativi e riducendo quindi l'efficacia complessiva nella rilevazione delle minacce.

%%%%%%%%%%%%% Inizio metriche quovadis %%%%%%%%%%%%%%%
%%%%%%%%%%%%% Random Forest %%%%%%%%%%%%%%%%%%%%%%%%

\begin{figure}[ht]
    \centering
    \adjustbox{max width=\linewidth, max height=0.8\textheight, keepaspectratio}{%
        \includesvg{validazione-empirica/imgs/mpasco_XGBoost_metrics.svg}
    }
    \caption{\clscaption{quovadis}{XGBoost}}
    \label{fig:mpasco-cls-xgb}
\end{figure}


\begin{figure}[ht]
    \centering
    \includesvg[width=\linewidth, keepaspectratio]{validazione-empirica/imgs/mpasco_RandomForest_metrics.svg}
    \caption{\clscaption{mpasco}{RandomForest}}
    \label{fig:mpasco-cls-rf}
\end{figure}

%%%%%%%%%%%%%%% FINE METRICHE CLASSE quovadis %%%%%%%%%%%%%%%%%%%%%%%%%%%%%%

Il classificatore \textit{RandomForest} ottiene ottimi risultati su tutte le classi.\
Il classificatore \textit{XGBoost} mostra buone prestazioni complessive, ma presenta un valore di \textit{recall}\
più basso per la classe \textit{Goodware}, evidenziando una difficoltà nel riconoscere correttamente un numero significativo di istanze appartenenti a tale classe.

\begin{figure}[t]
    \centering
    \adjustbox{max width=1.4\linewidth, max height=1.2\textheight}{%
        \includesvg[width=\textwidth]{validazione-empirica/imgs/quovadis-global-metrics.svg}
    }
    \caption{\grpdescription{quovadis}}
    \label{fig:octack-mtrx-rf}
\end{figure}

\FloatBarrier

Le metriche globali rispecchiano le prestazioni sulle classi ribadedno che il classificatore \textit{RandomForest}\
sia nettamente superiore su \textit{XGBoost}.

%%%%%%%%%%%%%%%%%%%%%%%%%%%%%%% Conclusioni %%%%%%%%%%%%%%%%%%%%%%%%%%%%%%%%%%%%%%

In conclusione, il classificatore \textit{RandomForest} si è dimostrato complessivamente più efficace rispetto a \textit{XGBoost},\
in particolare nella classificazione multiclasse dei campioni di malware, dove ha mostrato una maggiore accuratezza tra le diverse categorie.\
Entrambi i modelli hanno tuttavia raggiunto prestazioni elevate su tutti i dataset considerati.\
I risultati ottenuti sono stati tali da non rendere necessario un ulteriore processo di ottimizzazione degli iperparametri,\
indicando un buon bilanciamento tra complessità del modello e qualità della classificazione.
