\newcommand{\datasetapimds}{\textit{A Novel Approach to Detect Malware Based on API Call Sequence Analysis}}
\newcommand{\datasetoctak}{\textit{Data augmentation based malware detection using convolutional neural networks}}
\newcommand{\datasetmpasco}{\textit{API-MalDetect: Automated malware detection framework for windows based on API calls and deep learning techniques}}
\newcommand{\datasetquovaids}{\textit{Quo Vadis: Hybrid Machine Learning Meta-Model Based on Contextual and Behavioral Malware Representations}}
\newcommand{\clsper}[3]{\textit{#1} (\percc{#2} - $#3$)}
\section{Dataset}

La validazione empirica è stata condotta impiegando quattro dataset pubblici di chiamate API reperiti dalla letteratura scientifica.\
Nel seguito, essi verranno indicati mediante i seguenti alias: il dataset proposto da \datasetapimds\mycite{Ki_2015} sarà denominato \textbf{apimds};\
quello descritto in \datasetoctak\mycite{10.7717/peerj-cs.346} sarà indicato come \textbf{octak};\
il dataset introdotto da \datasetmpasco\mycite{maniriho2023api} come \textbf{mpasco}; infine, il dataset derivante da\
\datasetquovaids\mycite{10.1145/3560830.3563726} sarà indicato come \textbf{quovadis}.

%%%%%%%%%%%%%%%%%% APIMDS %%%%%%%%%%%%%%%%%%%%
Il dataset \textbf{apimds} è rappresentato in formato tabulare dove ogni riga (esempio)\
contiene nel seguente ordine le feature: la classe dell'esempio, l'hash del PE,\
e le restanti colonne le chiamate API effettuate.\
In particolare per l'addestramento è stata utilizzata la prima colonna (la classe) come etichetta e\
le colonne successive alla seconda per recuperare la sequenza di chiamate API.
In \autoref{fig:apimds-example} un esempio del dataset.\
Il numero totale di esempi è pari a $23146$, di cui \percc{80} dedicati al training ($18519$) e il restante \percc{20}($4627$) al testing.\
La media di chiamate API per esempio è pari a $284.391$ e il numero di chiamate distinte (quindi le feature) è pari a $1165$.\
Le classi presenti sono \clsper{Malware}{27.4}{6111}, \clsper{Backdoor}{2.6}{582}, \clsper{Downloader}{4,1}{917},\
\clsper{Trojan}{46.9}{10450}, \clsper{Virus}{14.6}{3216} e \clsper{Packed}{4.3}{964}.\
La classe \textit{Goodware} non è presente ed la classe \textit{Malware} è stata utilizzata per rappresentare tutti quegli esempi per\
i quali non è stato fornita un ulteriore specializzazione dal dataset.\
In \autoref{fig:apimds-classes-sets} e \autoref{fig:apimds-classes} vengono riportati le distribuzioni delle classi del dataset.\

\begin{figure}[h!]
    \centering
    \inputminted[fontsize=\small]{text}{validazione-empirica/files/apimds-example.csv}
    \caption{Una porzione del dataset \textbf{apimds} troncato per una maggiore leggibilità.}
    \label{fig:apimds-example}
\end{figure}

\begin{figure}[h!]
    \centering
    \adjustbox{max width=\linewidth, max height=\textheight}{%
        \includesvg{validazione-empirica/imgs/apimds-class-distribution.svg}%
    }
    \caption{Distribuzione delle classi sul dataset \textbf{apimds} per training set e test set}
    \label{fig:apimds-classes-sets}
\end{figure}


\begin{figure}[h!]
    \centering
    \adjustbox{max width=\linewidth, max height=\textheight}{%
        \includesvg{validazione-empirica/imgs/apimds-class_distribution.svg}%
    }
    \caption{Distribuzione delle classi sul dataset \textbf{apimds}}
    \label{fig:apimds-classes}
\end{figure}

\FloatBarrier

%%%%%%%%%%%%%%%%%% octak %%%%%%%%%%%%%%%%%%%%

Il dataset \textbf{octak} è disponibile su GitHub\mycite{ocatak_malware_api_class} ed è distribuito in formato tabulare su due tabelle.\
La prima tabella, dove ogni riga corrisponde un esempio, ha una sola feature che corrisponde alla classe dell'esempio.\
La seconda tabella, dove ogni riga corrisponde un esempio, ha una sola colonna in cui è presenta la sequenza di chiamate API\
separate da spazio.\
In particolare, per l'addestramento, si è utilizzata l'$i$-esima riga della prima tabella per ricavare la classe\
e l'$i$-esima riga della seconda tabella per ricavare la sequenza di chiamate API per l'$i$-esimo esempio.\
In \autoref{fig:octck-example} un esempio del dataset.\
Il numero di esempi è pari a $7107$, di cui \percc{80} dedicato al training ($56880$) e il restante \percc{20} al testing set ($1419$).\
La media di chiamate API per esempio è pari a $19647.267$ e il numero di chiamate distinte è pari a $278$.\
Le classi presenti sono \clsper{Backdoor}{14.2}{1001}, \clsper{Adware}{5.4}{379}, \clsper{Downloader}{14.2}{1001}, \clsper{Dropper}{12.6}{891},\
\clsper{Spyware}{11.8}{832}, \clsper{Trojan}{14.2}{1001}, \clsper{Virus}{14.2}{1001}, \clsper{Worm}{13.6}{972}.\
Non sono presenti istanze di \textit{Goodware}.
In \autoref{fig:octak-classes-sets} e \autoref{fig:octak-classes} vengono riportati le distribuzioni delle classi del dataset.

\begin{figure}[htbp]
    \centering
    % Minipage sinistra (20%)
    \begin{minipage}[t]{0.2\textwidth}
        \centering
        \inputminted[fontsize=\small]{text}{validazione-empirica/files/octak-labels.txt}
    \end{minipage}%
    \hfill
    % Minipage destra (80%)
    \begin{minipage}[t]{0.75\textwidth} % lascia un piccolo gap con hfill
        \centering
        \inputminted[fontsize=\small]{text}{validazione-empirica/files/octak-calls.csv}
        % caption separata
    \end{minipage}
    \caption{Una porzione del dataset \textbf{octack} troncato per maggior leggibilità.}
    \label{fig:octck-example}
\end{figure}

\begin{figure}[h!]
    \centering
    \adjustbox{max width=\linewidth, max height=\textheight}{%
        \includesvg{validazione-empirica/imgs/octak-class-distribution.svg}%
    }
    \caption{Distribuzione delle classi sul dataset \textbf{octak} per training set e test set}
    \label{fig:octak-classes-sets}
\end{figure}

\begin{figure}[h!]
    \centering
    \adjustbox{max width=\linewidth, max height=\textheight}{%
        \includesvg{validazione-empirica/imgs/octak-class_distribution.svg}%
    }
    \caption{Distribuzione delle classi sul dataset \textbf{octak}}
    \label{fig:octak-classes}
\end{figure}

\FloatBarrier

%%%%%%%%%%%%%%%%%% mpasco %%%%%%%%%%%%%%%%%%%%

Il dataset \textbf{mpasco} è disponibile su GitHub\mycite{maniriho2023malbehavd}, distribuito in formato tabulare,\
dove ogni riga è un esempio e in ordine troviamo le seguenti feature: sha265, labels, e una sequenza incrementale\
che indica la posizione della chiamata API nella sequenza.\
Per l'addestramento le feature selezionate sono state la label (con valore $0$ per \textit{Goodware} altrimenti \textit{Malware}),\
e le colonne della sequenza della chiamate API.\
In \autoref{fig:mpasco-example} un esempio del dataset.\
Il numero di esempi è pari a $4112$, di cui \percc{80} dedicato al training ($3598$) e il restante \percc{20} al testing set ($514$).\
La media di chiamate API per esempio è pari a $43.187$ e il numero di chiamate distinte è pari a $291$.\
Sono presenti solo due classi, \clsper{Malware}{50}{2056} e \clsper{Goodware}{50}{20560}.
In \autoref{fig:mpasco-classes-sets} e \autoref{fig:mpasco-classes} vengono riportati le distribuzioni delle classi del dataset.

\begin{figure}[h!]
    \centering
    \inputminted[fontsize=\small]{text}{validazione-empirica/files/mpasco-example.csv}
    \caption{Una porzione del dataset \textbf{mpasco} troncato per una maggiore leggibilità.}
    \label{fig:mpasco-example}
\end{figure}


\begin{figure}[h!]
    \centering
    \adjustbox{max width=\linewidth, max height=\textheight}{%
        \includesvg{validazione-empirica/imgs/mpasco-class-distribution.svg}%
    }
    \caption{Distribuzione delle classi sul dataset \textbf{mpasco} per training set e test set}
    \label{fig:mpasco-classes-sets}
\end{figure}

\begin{figure}[h!]
    \centering
    \adjustbox{max width=\linewidth, max height=\textheight}{%
        \includesvg{validazione-empirica/imgs/mpasco-class_distribution.svg}%
    }
    \caption{Distribuzione delle classi sul dataset \textbf{mpasco}}
    \label{fig:mpasco-classes}
\end{figure}

\FloatBarrier

%%%%%%%%%%%%%%%%%% quovadis %%%%%%%%%%%%%%%%%%%%

Il dataset \textbf{quovadis} è disponibile su GitHub\mycite{trizna_quovadis} ed è distribuito come un archivio compresso.\
Nell'archivio sono presenti diverse cartelle che rispettano la denominazione \texttt{report\_\textless classe\textgreater},\
dove \texttt{classe} è la classe delle istanze degli esempi presenti.\
In una singola cartella di classe, sono presenti molteplici file JSON con la denominazione \texttt{\textless hash PE\textgreater.json}.
I campi di interesse del JSON per la standardizzazione del dataset sono riportati in tabella:

\vspace{0.5cm} % aggiunge mezzo centimetro di spazio

\begin{tabular}[H]{p{0.25\textwidth} p{0.65\textwidth}}
    \toprule
    \textbf{Key}                 & \textbf{Descrizione}                                                \\
    \midrule
    \texttt{[].apis}             & Un array di oggetti JSON rappresentanti le chiamate API in sequenza \\
    \texttt{[].apis[].api\_name} & La chiamata API                                                     \\
    \bottomrule
    \label{fig:quovadis-classes-sets}
\end{tabular}


\FloatBarrier
In \autoref{fig:quovadis-example} un esempio del dataset.\
Il numero di esempi è pari a $64023$, di cui \percc{80} dedicato al training ($51219$) e il restante \percc{20} al testing set ($12804$).\
La media di chiamate API per esempio è pari a $196.144$ e il numero di chiamate distinte è pari a $3024$.\
Sono presenti tre classi, \clsper{Malware}{43.2}{27670}, \clsper{Goodware}{39.5}{25291} e \clsper{Backdoor}{17.3}{11062}.\
Nella classe malware sono racchiusi tutti i malware non specializzati.
In \autoref{fig:quovadis-classes-sets} e \autoref{fig:quovadis-classes} vengono riportati le distribuzioni delle classi del dataset.

\begin{figure}[h!]
    \centering
    \inputminted[fontsize=\small]{json}{validazione-empirica/files/quovadis-example.json}
    \caption{Una porzione del dataset \textbf{quovadis} troncato per una maggiore leggibilità.}
    \label{fig:quovadis-example}
\end{figure}


\begin{figure}[h!]
    \centering
    \adjustbox{max width=\linewidth, max height=\textheight}{%
        \includesvg{validazione-empirica/imgs/quovadis-class-distribution.svg}%
    }
    \caption{Distribuzione delle classi sul dataset \textbf{quovadis} per training set e test set}
    \label{fig:quovadis-classes-sets}
\end{figure}

\begin{figure}[h!]
    \centering
    \adjustbox{max width=\linewidth, max height=\textheight}{%
        \includesvg{validazione-empirica/imgs/quovadis-class_distribution.svg}%
    }
    \caption{Distribuzione delle classi sul dataset \textbf{quovadis}}
    \label{fig:quovadis-classes}
\end{figure}

\FloatBarrier

In \autoref{tab:dataset-summary} la tabella riassuntiva dei $4$ dataset.

\begin{table}[h!]
    \centering
    \begin{tabular}{|l|l|r|r|r|}
        \hline
        \textbf{Dataset} & \textbf{Classificazione} & \textbf{Numero feature} & \textbf{Esempi train} & \textbf{Esempi test} \\
        \hline
        apimds           & Multiclasse              & 1165                    & 18519                 & 4627                 \\
        octak            & Multiclasse              & 278                     & 56880                 & 1419                 \\
        mpasco           & Binaria                  & 291                     & 3598                  & 514                  \\
        mpasco           & Multiclasse              & 3024                    & 51219                 & 12804                \\
        \hline
    \end{tabular}
    \caption{Caratteristiche dei dataset utilizzati.}
    \label{tab:dataset-summary}
\end{table}

\FloatBarrier
