\newcommand{\datasetapimds}{\textit{A Novel Approach to Detect Malware Based on API Call Sequence Analysis}}
\newcommand{\datasetoctak}{\textit{Data augmentation based malware detection using convolutional neural networks}}
\newcommand{\datasetmpasco}{\textit{API-MalDetect: Automated malware detection framework for windows based on API calls and deep learning techniques}}
\newcommand{\datasetquovaids}{\textit{Quo Vadis: Hybrid Machine Learning Meta-Model Based on Contextual and Behavioral Malware Representations}}
\newcommand{\clsper}[3]{\textit{#1} (\percc{#2} - $#3$)}
\section{Dataset}

La validazione empirica è stata condotta impiegando quattro dataset pubblici reperiti dalla letteratura scientifica.\
Nel seguito, essi verranno indicati mediante i seguenti alias: il dataset proposto da \datasetapimds\mycite{Ki_2015} sarà denominato \textbf{apimds};\
quello descritto in \datasetoctak\mycite{10.7717/peerj-cs.346} sarà indicato come \textbf{octak};\
il dataset introdotto da \datasetmpasco\mycite{maniriho2023api} come \textbf{mpasco}; infine, il dataset derivante da\
\datasetquovaids\mycite{10.1145/3560830.3563726} sarà indicato come \textbf{quovadis}.

%% Per ogni dataser riportare se è bilanciato o meno, 
%%Aggiungi dati che diano un'idea della scala:
%%
%%Numero totale di campioni
%%
%%Numero totale di classi
%%
%%Numero medio di chiamate API per campione
%%
%%Numero minimo/massimo di chiamate API
%%
%% Dimensione del training e del test set (in numero di istanze e in percentuale)
Il dataset \textbf{apimds} è distribuito come file CSV denominato \texttt{malware\_API\_dataset.csv}.\
Le colonne del CSV rappresentano in ordine, l'hash del PE, la classe di riferimento e le chiamate API effettuate.\
Per la standardizzazione solo i dati dalla seconda colonna in poi sono stati utilizzati.\
Il numero totale di esempi è pari a $23146$, di cui \percc{80} dedicati al training ($18519$) e il restante \percc{20}($4627$) al testing.\
La media di chiamate API per esempio è pari a $284.39199$ e il numero di chiamate distinte (quindi le feature) è pari a $1165$.
Le classi presenti sono \clsper{Malware}{27.4}{6111}, \clsper{Backdoor}{2.6}{582}, \clsper{Downloader}{4,1}{917},\
\clsper{Trojan}{46.9}{10450}, \clsper{Virus}{14.6}{3216} e \clsper{Packed}{4.3}{964}.
La classe \textit{Goodware} non è presente ed la classe \textit{Malware} raggruppa tutti i malware non specificati nel dataset.\
In \autoref{fig:apimds-classes-sets} e \autoref{fig:apimds-classes} vengono riportati le distribuzioni delle classi del dataset.

\begin{figure}[h!]
    \centering
    \adjustbox{max width=\linewidth, max height=\textheight}{%
        \includesvg{validazione-empirica/imgs/apimds-class-distribution.svg}%
    }
    \caption{Distribuzione delle classi sul dataset \textbf{apimds} per training set e test set}
    \label{fig:apimds-classes-sets}
\end{figure}


\begin{figure}[h!]
    \centering
    \adjustbox{max width=\linewidth, max height=\textheight}{%
        \includesvg{validazione-empirica/imgs/apimds-class_distribution.svg}%
    }
    \caption{Distribuzione delle classi sul dataset \textbf{apimds}}
    \label{fig:apimds-classes}
\end{figure}

\FloatBarrier

Il dataset \textbf{octak} è disponibile su GitHub\mycite{ocatak_malware_api_class} ed è distribuito usando un file CSV e uno di testo.\
Il file \texttt{label.csv} è composto da una sola colonna contenente la classe dell'esempio; Il file \texttt{all\_analysis\_data.txt},\
contiene le rispettive sequenze di API calls separate da spazi per riga.\
Per la conversione nel formato standardizzato, sono state unite le informazioni dei due file associando l'$i$-esima riga del file \texttt{label.csv}\
con l'$i$-esima riga del file \texttt{all\_analysis\_data.txt}.\
Il numero di esempi è pari a $7107$, di cui \percc{80} dedicato al training ($56880$) e il restante \percc{20} al testing set ($1419$).\
Le classi presenti sono \clsper{Backdoor}{14.2}{1001}, \clsper{Adware}{5.4}{379}, \clsper{Downloader}{14.2}{1001}, \clsper{Dropper}{12.6}{891},\
\clsper{Spyware}{11.8}{832}, \clsper{Trojan}{14.2}{1001}, \clsper{Virus}{14.2}{1001}, \clsper{Worm}{13.6}{972}.\
Non sono presenti instanze di \textit{Goodware}.
In \autoref{fig:octak-classes-sets} e \autoref{fig:octak-classes} vengono riportati le distribuzioni delle classi del dataset.

\begin{figure}[h!]
    \centering
    \adjustbox{max width=\linewidth, max height=\textheight}{%
        \includesvg{validazione-empirica/imgs/octak-class-distribution.svg}%
    }
    \caption{Distribuzione delle classi sul dataset \textbf{octak} per training set e test set}
    \label{fig:octak-classes-sets}
\end{figure}

\begin{figure}[h!]
    \centering
    \adjustbox{max width=\linewidth, max height=\textheight}{%
        \includesvg{validazione-empirica/imgs/octak-class_distribution.svg}%
    }
    \caption{Distribuzione delle classi sul dataset \textbf{octak}}
    \label{fig:octak-classes}
\end{figure}

\FloatBarrier

Il dataset \textbf{mpasco} è disponibile su GitHub\mycite{maniriho2023malbehavd}, distribuito come file CSV \texttt{MalBehavD-V1-dataset.csv}.\
Le colonne del CSV rappresentano in ordine l'hash del PE, la classe dell'esempio ($0$ per \textit{Goodware} altrimenti \textit{Malware}) e\
le successive colonne la sequenza di chiamate API. Per la standardizzazione del dataset, solo la prima colonna è stata esclusa.
Il numero di esempi è pari a $4112$, di cui \percc{80} dedicato al training ($3598$) e il restante \percc{20} al testing set ($514$).\
Sono presenti solo due classi, \clsper{Malware}{50}{2056} e \clsper{Goodware}{50}{20560}.
In \autoref{fig:mpasco-classes-sets} e \autoref{fig:mpasco-classes} vengono riportati le distribuzioni delle classi del dataset.

\begin{figure}[h!]
    \centering
    \adjustbox{max width=\linewidth, max height=\textheight}{%
        \includesvg{validazione-empirica/imgs/mpasco-class-distribution.svg}%
    }
    \caption{Distribuzione delle classi sul dataset \textbf{mpasco} per training set e test set}
    \label{fig:mpasco-classes-sets}
\end{figure}

\begin{figure}[h!]
    \centering
    \adjustbox{max width=\linewidth, max height=\textheight}{%
        \includesvg{validazione-empirica/imgs/mpasco-class_distribution.svg}%
    }
    \caption{Distribuzione delle classi sul dataset \textbf{mpasco}}
    \label{fig:mpasco-classes}
\end{figure}

\FloatBarrier

Il dataset \textbf{quovadis} è disponibile su GitHub\mycite{trizna_quovadis} ed è distribuito come un archivio compresso.\
Nell'archivio sono presenti diverse cartelle che rispettano la denominazione \texttt{report\_\textless classe\textgreater},\
dove \texttt{classe} è la classe delle istanze degli esempi presenti.\
In una singola cartella di classe, sono presenti molteplici file JSON con la denominazione \texttt{\textless hash PE\textgreater.json}.
I campi di interesse del JSON per la standardizzazione del dataset sono riportati in tabella:

\vspace{0.5cm} % aggiunge mezzo centimetro di spazio

\begin{tabular}[H]{p{0.25\textwidth} p{0.65\textwidth}}
    \toprule
    \textbf{Key}                 & \textbf{Descrizione}                                                \\
    \midrule
    \texttt{[].apis}             & Un array di oggetti JSON rappresentanti le chiamate API in sequenza \\
    \texttt{[].apis[].api\_name} & La chiamata API                                                     \\
    \bottomrule
    \label{fig:quovadis-classes-sets}
\end{tabular}


\FloatBarrier

Il numero di esempi è pari a $64023$, di cui \percc{80} dedicato al training ($51219$) e il restante \percc{20} al testing set ($12804$).\
Sono presenti tre classi, \clsper{Malware}{43.2}{27670}, \clsper{Goodware}{39.5}{25291} e \clsper{Backdoor}{17.3}{11062}.\
Nella classe malware sono racchiusi tutti i malware non specializzati.
In \autoref{fig:quovadis-classes-sets} e \autoref{fig:quovadis-classes} vengono riportati le distribuzioni delle classi del dataset.

\begin{figure}[h!]
    \centering
    \adjustbox{max width=\linewidth, max height=\textheight}{%
        \includesvg{validazione-empirica/imgs/quovadis-class-distribution.svg}%
    }
    \caption{Distribuzione delle classi sul dataset \textbf{quovadis} per training set e test set}
    \label{fig:quovadis-classes-sets}
\end{figure}

\begin{figure}[h!]
    \centering
    \adjustbox{max width=\linewidth, max height=\textheight}{%
        \includesvg{validazione-empirica/imgs/quovadis-class_distribution.svg}%
    }
    \caption{Distribuzione delle classi sul dataset \textbf{quovadis}}
    \label{fig:quovadis-classes}
\end{figure}

\FloatBarrier