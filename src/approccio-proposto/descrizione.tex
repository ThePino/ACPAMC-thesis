\section{Descrizione}

La fase sperimentale di questa tesi si è concentrata sulla \textbf{classificazione di software} attraverso
l'analisi dei pattern comportamentali espressi dalle sequenze di chiamate API.\
Il percorso metodologico seguito può essere suddiviso nelle seguenti fasi principali:\
la raccolta e la standardizzazione dei dati, l'ingegneria delle caratteristiche, l'addestramento dei classificatori
e la valutazione finale.

\subsection{Raccolta e Standardizzazione dei Dati}

Per l'analisi sono stati raccolti \textbf{quattro diversi dataset} da fonti esterne.\
In origine, questi dati erano disponibili in formati eterogenei\
(ad esempio CSV\footnote{Il formato \textbf{CSV} (Comma Separated Value) è un formato di condivisione dati in cui ogni
    riga rappresenta un dato di cui le caratteristiche sono separate da virgola \mycite{Shafranovich_2005}.},
JSON\footnote{Il formato \textbf{JSON} (JavaScript Object Notation) è un formato per lo scambio di dati\
    basato su una collezione di coppie nome/valore e di array\mycite{Bray_2017}.}, e formati personalizzati),\
rendendo difficoltosa un'analisi congiunta e omogenea.\
Per superare questa limitazione, si è resa necessaria una fase di \textbf{standardizzazione} in un unico formato:
il \textbf{JSON}.\
La struttura adottata per ogni file standardizzato è stata pensata per rappresentare in modo chiaro\
e minimale le informazioni essenziali per l'addestramento.\
Lo schema per il JSON scelto è illustrato in \autoref{fig:json-struct}:

\begin{figure}[h!]
    \centering
    \renewcommand{\arraystretch}{1.3}
    \begin{tabular}{p{0.25\textwidth} p{0.65\textwidth}}
        \toprule
        \textbf{Key}                  & \textbf{Descrizione}                                           \\
        \midrule
        \texttt{[].application\_type} & Classe del software (\emph{malware}, \emph{goodware}, $\dots$) \\
        \texttt{[].apis}              & Sequenza di chiamate API (array di stringhe)                   \\
        \bottomrule
    \end{tabular}

    \vspace{4mm} % spazio tra tabella e codice

    \inputminted[fontsize=\small]{json}{approccio-proposto/example.json}
    \caption{Descrizione dei campi e corrispondente esempio JSON.}
    \label{fig:json-struct}
\end{figure}

\subsection{Ingegneria delle Caratteristiche: Il Modello Bag-of-Words}

L'obiettivo della classificazione è distinguere le categorie di software basandosi sul loro comportamento,\
espresso dalle sequenze di chiamate API.\
Per tradurre queste sequenze in un formato numerico interpretabile dagli algoritmi di Machine Learning,\
è stato costituito un \textbf{vettore delle caratteristiche} (feature vector) basato sulle\
\textbf{frequenze} con cui ogni singola API compare nella sequenza.\
A tal fine, è stato adottato l'approccio \textbf{Bag-of-Words} (BoW).\
In letteratura, questa tecnica, presente nel Natural Language Processing,\
consente di rappresentare un documento testuale (nel nostro caso, la sequenza di API di un programma)
come un vettore numerico che ne riflette il contenuto, ignorando l'ordine temporale con cui le ``parole''\
(ovvero, le singole API) compaiono all'interno della sequenza\mycite{Wikipedia_contributors_2025}.\
Per implementare il modello BoW e ottenere il vettore delle caratteristiche, è stata eseguita la seguente procedura:

\begin{enumerate}
    \item \textbf{Creazione del Vocabolario:} Tutte le chiamate API univoche presenti nel dataset\
          sono state estratte per comporre il vocabolario.
    \item \textbf{Indicizzazione:} Il vocabolario è stato \textbf{ordinato alfabeticamente}.\
          Tale ordinamento stabilisce una corrispondenza univoca e fissa: la posizione $i$-esima nel vocabolario\
          (data dall'ordine alfabetico) corrisponde alla posizione $i$-esima nel vettore delle caratteristiche.
    \item \textbf{Vettorizzazione:} Per ogni esempio si crea un vettore di interi avente cardinalità pari alla\
          dimensione del vocabolario, con tutti i valori inizialmente pari a zero.
    \item \textbf{Popolamento:} Per ogni chiamata API presente nella sequenza del esempio,\
          si individua la sua posizione nel vocabolario (grazie all'indicizzazione) e\
          si incrementa di $1$ il valore nella corrispondente posizione del vettore delle caratteristiche.
\end{enumerate}

Il risultato finale è un vettore di frequenze assolute a lunghezza fissa,\
in cui ogni componente quantifica il numero di occorrenze di una specifica API.

\subsection{Addestramento e Valutazione}

Ottenuti i vettori delle caratteristiche, si è passati alla fase di addestramento.\
L'implementazione è stata realizzata utilizzando la libreria \textit{Scikit-learn}\mycite{scikit-learn},\
che offre i modelli \textit{XGBoost} e \textit{RandomForest}.\
Il processo ha previsto la suddivisione del dataset originario in due istanze:
\begin{itemize}
    \item \textbf{Training Set (\percc{80})}: utilizzato per addestrare il modello tramite il metodo \textit{fit}.
    \item \textbf{Evaluation Set (\percc{20})}: utilizzato per la valutazione delle prestazioni tramite il metodo \textit{predict}.
\end{itemize}

Per la divisione è stata applicata la tecnica di \textbf{Campionamento Stratificato} (Stratified Sampling)\mycite{Wikipedia_campionamento_stratificato}.\
Tale scelta metodologica è stata essenziale per garantire la \textbf{proporzionalità} di presenza di tutte le classi\
di software in entrambi i sottoinsiemi (training e valutazione).\
In questo modo si è prevenuta la distorsione del modello, garantendo che anche le classi minoritarie\
fossero adeguatamente rappresentate nel processo di apprendimento e nella successiva fase di testing.

Infine, nella fase di valutazione, le previsioni effettuate dal metodo \textit{predict} sono state utilizzate\
per calcolare le metriche di interesse, partendo dalla costruzione della \textbf{Matrice di Confusione}.