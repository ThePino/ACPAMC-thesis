\section{OOP \& UML}

La fase di implementazione del progetto è stata condotta adottando il paradigma della \textbf{Programmazione Orientata agli Oggetti} (OOP).\
Questo approccio si è rivelato fondamentale per strutturare il codice in modo \textit{modulare}, \textit{manutenibile} e \textit{scalabile},\
in quanto ha permesso di modellare le entità del problema (come i dataset, le sequenze di API e i classificatori) come componenti software interconnesse.\
L'efficacia dell'OOP si fonda su quattro pilastri concettuali:
%... (Lascia qui l'itemize con i 4 pilastri) ...
\begin{itemize}
    \item \textit{Astrazione}: Consente di modellare le entità come classi, focalizzandosi unicamente sugli attributi e sulle interazioni rilevanti per il contesto applicativo,
          nascondendo la complessità sottostante \mycite{oop}.
    \item \textit{Incapsulamento}: Protegge lo stato interno di un oggetto, nascondendolo e consentendo l'interazione con le sue funzionalità per garantirne l'integrità\mycite{oop}.
    \item \textit{Ereditarietà}: Permette di creare nuove astrazioni (sottoclassi) basandosi su astrazioni esistenti (superclassi),
          facilitando il riuso del codice e stabilendo gerarchie logiche tra le classi \mycite{oop}.
    \item \textit{Polimorfismo}: Consente a oggetti diversi di rispondere allo stesso messaggio (chiamata di metodo) in modi specifici alla propria classe,
          permettendo di implementare proprietà o metodi ereditati in forme distinte tra diverse astrazioni \mycite{oop}.
\end{itemize}

Per formalizzare e documentare questa architettura basata sui principi OOP, è stato utilizzato l'\textbf{Unified Modeling Language} (UML)\mycite{ibm_uml}.\
Nello specifico, il \textbf{Diagramma delle Classi} è stato scelto per illustrare la struttura del sistema e le relazioni tra i suoi componenti.\
Il diagramma delle classi completo che definisce l'architettura logica del progetto è presentato in \autoref{fig:pdfdoc}.\
Far riferimento alla \autoref{sec:convenzioni-uml} per una guida su come leggere il diagramma UML.

\begin{figure}[htbp] % h=qui, t=top, b=bottom, p=pagina dedicata
    \centering
    % Inserisci la PRIMA pagina del PDF come immagine
    \adjustbox{max width=\linewidth, max height=\textheight}{%
        \includesvg[angle=90]{approccio-proposto/imgs/uml_svg.svg}%
    }
    \caption{Diagramma delle classi UML che descrive l'architettura OOP del sistema.}
    \label{fig:pdfdoc}
\end{figure}

\FloatBarrier

L'intero sistema è basato su $3$ package: \textit{data}, \textit{classifier} e \textit{eval}.

Il package \textit{data} gestisce tutte le entità del dominio, dal caricamento dei dati in memoria fino alla loro rappresentazione vettoriale per l'addestramento dei classificatori.\
Le classi \textit{ApiCall} e \textit{Sequence} descrivono la traccia dinamica di un file PE: la prima incapsula una singola chiamata API,\
mentre la seconda raccoglie una sequenza ordinata di chiamate.\
Per aggiungere l'informazione relativa alla classe di appartenenza di una sequenza, il package introduce l'enumerazione \textit{ApplicationType},\
che elenca le possibili categorie (classi) del dominio, e la classe \textit{LabeledSequence}, che associa ogni sequenza alla propria classe.\
La gestione complessiva del dataset è affidata alla classe \textit{LabeledSequenceDataset}, che raccoglie e organizza un insieme di sequenze etichettate,\
occupandosi anche del caricamento dei dati in formato JSON e della logica di suddivisione del dataset.\
Infine, per rappresentare i dati in una forma adatta agli algoritmi di classificazione, sono state introdotte le classi \textit{FeatureVector}, \textit{LabeledFeatureVector} e \textit{LabeledFeatureVectorDataset},
che modellano rispettivamente un singolo vettore di caratteristiche, un vettore con la relativa etichetta e l'intero insieme dei vettori etichettati.

\begin{figure}[htbp] % h=qui, t=top, b=bottom, p=pagina dedicata
    \centering
    % Inserisci la PRIMA pagina del PDF come immagine
    \adjustbox{max width=\linewidth, max height=\textheight}{%
        \includesvg[]{approccio-proposto/imgs/data.svg}%
    }
    \caption{UML - Package data}
    \label{fig:pacakge-data}
\end{figure}

Il package \textit{classifier} rappresenta l'astrazione e le sue specializzazioni per un classificatore.\
La classe astratta \textit{Classifier} ingloba i metodi \textit{predict} e \textit{fit}, responsabili rispettivamente della classificazione e dell'allenamento del modello.\
Le classi \textit{RandomForest} e \textit{XGBoost} specializzano la classe \textit{Classifier}, implementando gli omonimi algoritmi di classificazione.

\begin{figure}[htbp] % h=qui, t=top, b=bottom, p=pagina dedicata
    \centering
    % Inserisci la PRIMA pagina del PDF come immagine
    \adjustbox{max width=\linewidth, max height=\textheight}{%
        \includesvg[]{approccio-proposto/imgs/classifier.svg}%
    }
    \caption{UML - Package classifier}
    \label{fig:pacakge-classifier}
\end{figure}

Il package \textit{eval} è responsabile delle metriche di valutazione del modello.\
La classe \textit{Evaluator} implementa le principali metriche di valutazione, includendo al suo interno la matrice di confusione.\
La classe \textit{Metric} è un'enumerazione utilizzata per parametrizzare i metodi di calcolo delle metriche nella classe \textit{Evaluator}.

\FloatBarrier

\begin{figure}[htbp] % h=qui, t=top, b=bottom, p=pagina dedicata
    \centering
    % Inserisci la PRIMA pagina del PDF come immagine
    \adjustbox{max width=\linewidth, max height=\textheight}{%
        \includesvg[]{approccio-proposto/imgs/eval.svg}%
    }
    \caption{UML - Package eval}
    \label{fig:pacakge-eval}
\end{figure}