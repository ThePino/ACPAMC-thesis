\chapter{Conclusioni e Sviluppi Futuri}

In questo capitolo conclusivo vengono riassunti i risultati ottenuti dal lavoro di ricerca e sviluppo svolto,\
evidenziando il raggiungimento degli obbiettivi preposti.\
Successivamente, vengono delineate le possibili direzioni future per estendere e migliorare l'approccio proposto,\
con particolare attenzione all'integrazione dei dati e all'adozione di architetture di apprendimento più complesse.

\section{Conclusioni}

Il presente lavoro di tesi ha affrontato il problema della classificazione di malware in ambiente Windows attraverso l'analisi dinamica e l'utilizzo di tecniche di Machine Learning.
L'obbiettivo principale era sviluppare un sistema in grado di distinguere non solo tra software benino e malevolo, ma anche di categorizzare le diverse famiglie di \textit{Malware},
basandosi esclusivamente sui pattern comportamentali espressi dalle sequenze di chiamate API.\
L'approccio proposto ha previsto la standardizzazione di dataset eterogenei (\textit{apimds}, \textit{octack}, \textit{mpasco}, \textit{quovadis}) e l'implementazione di un modello\
attraverso la rappresentazione \textit{Bag-of-Words} per la trasformazione delle sequenze in vettori delle caratteristiche.\
La validazione empirica è stata condotta confrontando gli algoritmi \textit{Random Forest} e \textit{XGBoost}.

Dall'analisi dei risultati sperimentali, emerge che l'efficacia dell'analisi dinamica: l'utilizzo delle sole frequenze di chiamate API si è dimostrato sufficiente per ottenere\
elevate accuratezze di classificazione, confermando l'ipotesi che i malware tendano a utilizzare sottoinsiemi specifici e ricorrenti di funzionalità del sistema operativo per raggiungere i propri scopi.\
Nel confronto tra i due classificatori, \textit{RandomForest} si è dimostrato complessivamente più robusto e stabile rispetto a \textit{XGBoost}.\
Tale superiorità è risultata più particolarmente evidente negli scenari multi-classe e in presenza di dataset sbilanciati (\textit{octack} e \textit{apimds}),\
dove \textit{RandomForest} ha garantito una migliore generalizzazione, mantenendo valori alti di \textit{f1-score} anche sulle classi minoritarie a differenza di \textit{XGBoost}.

È importante sottolineare che questi ottimi risultati sono stati raggiunti utilizzando gli algoritmi nella loro configurazione standard.\
Questa evidenza ha reso superfluo un oneroso processo di ottimizzazione degli iperparametri, indicando che la qualità delle feature estratte è il fattore determinante per il successo della classificazione.\
In sintesi, il sistema sviluppato ha dimostrato la capacità di identificare le minacce con elevata precisione osservando il comportamento del software piuttosto che la sua struttura statica, superando così i limiti delle tecniche di offuscamento che affliggono l'analisi statica tradizionale.

\section{Sviluppi Futuri}
