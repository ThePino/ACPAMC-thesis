\PassOptionsToPackage{table, dvipsnames}{xcolor}

\documentclass[a4paper,twoside,12pt]{toptesi}

%-------------------------------------------------------------------------------
% CONFIGURAZIONE BASE DEL DOCUMENTO
%-------------------------------------------------------------------------------
\usepackage[T1]{fontenc}
\usepackage[utf8]{inputenc}
\usepackage[italian]{babel}
\usepackage{amssymb}

%-------------------------------------------------------------------------------
% PACKAGE PER GRAFICA E IMMAGINI
%-------------------------------------------------------------------------------
\usepackage{multirow}
\usepackage{geometry}
\geometry{a4paper, margin=1in}
\usepackage{graphicx}
\usepackage{tikz}
\usetikzlibrary{positioning}
\usepackage{listings}
\usepackage[font=small,skip=10pt]{caption}
\usepackage{subcaption}
\usepackage{booktabs} 
\usepackage{longtable} 
\usepackage{float} 
\usepackage{caption} 
\usepackage{siunitx} 
\usepackage{makecell}

%-------------------------------------------------------------------------------
% PACKAGE PER COLORI E TABELLE
%-------------------------------------------------------------------------------

\usepackage[table]{xcolor}
\usepackage{booktabs}

%-------------------------------------------------------------------------------
% PACKAGE PER CODICE SORGENTE
%-------------------------------------------------------------------------------
\usepackage{minted}
\usepackage{amssymb} % if not already loaded

\setminted{
  style=friendly,
  linenos=true,
  frame=lines,
  framesep=2mm,
  bgcolor=gray!10,
  fontsize=\small,
  autogobble=true,
  breaklines=true,
  breakanywhere=true,
  breakindent=15pt,
  breakautoindent=true,
  breaksymbolleft=\mbox{\tiny\color{red}{\ensuremath{\Lsh}}\space},
}

%-------------------------------------------------------------------------------
% PACKAGE PER MATEMATICA E UNITÀ DI MISURA
%-------------------------------------------------------------------------------
\usepackage{siunitx}

%-------------------------------------------------------------------------------
% PACKAGE PER LISTE
%-------------------------------------------------------------------------------
\usepackage{enumitem}

%-------------------------------------------------------------------------------
% PACKAGE PER IPERTESTI E RIFERIMENTI
%-------------------------------------------------------------------------------
\usepackage{hyperref}
\hypersetup{
    pdfkeywords={Malware Detection, Fourier-BERT, Machine Learning, Cybersecurity, NLP},
    colorlinks=true,
    linkcolor=black,
    citecolor=Green,
    urlcolor=NavyBlue,
    filecolor=magenta,
    bookmarksnumbered=true,
    breaklinks=true,
    pdfpagemode=UseOutlines,
    pdfstartview=FitH,
    unicode=true
}

%-------------------------------------------------------------------------------
% IMPOSTAZIONI DI FORMATTAZIONE
%-------------------------------------------------------------------------------
\linespread{1.25}

\let\savedlisting\listing% save original definition
\AtBeginDocument{\let\listing\savedlisting}

%-------------------------------------------------------------------------------
% DEFINIZIONI PER IL FRONTESPIZIO (MACRO PERSONALIZZATE DALLA CLASSE toptesi)
%-------------------------------------------------------------------------------
\def\dept{DIPARTIMENTO DI INFORMATICA}
\def\course{CORSO DI LAUREA IN INFORMATICA}

\def\title{Analisi Dei Pattern Delle Chiamate Api Per La Classificazione Malware}
\def\author{Giacomo Gaudio}
\def\relatoreone{Prof.ssa Annalisa Appice}
\def\relatoretwo{}
\def\correlatoreone{Dott.ssa Giuseppina Andresini}
\def\correlatoretwo{}
\def\subject{}
\def\annoacc{2024 \- 2025}
\def\beforecandidate{LAUREANDO:}
\def\beforetitle{TESI DI LAUREA \\ IN \\ }
\def\beforeprof{RELATORE:}
\def\beforecorrelatore{CORRELATORI:}
\def\beforeannoacc{ANNO ACCADEMICO}

%-------------------------------------------------------------------------------
% DEFINIZIONE MACRO PERSONALIZZATE
%-------------------------------------------------------------------------------
\makeatletter
\def\cleardoublepage{\clearpage\if@twoside \ifodd\c@page\else
    \hbox{}
    \vspace*{\fill}
    \vspace{\fill}
    \thispagestyle{empty}
    \newpage
    \if@twocolumn\hbox{}\newpage\fi\fi\fi}
\makeatother

%-------------------------------------------------------------------------------
% INIZIO DEL DOCUMENTO
%-------------------------------------------------------------------------------
\begin{document}

%-------------------------------------------------------------------------------
% FRONTESPIZIO
%-------------------------------------------------------------------------------
\begin{titlepage}
    \begin{tikzpicture}[remember picture,overlay]
        \centering
        \node[yshift=-6 cm] (logo) at (current page.north) {\includegraphics[width=0.75\linewidth]{uniba.jpg}};
        \node[text width=50em,yshift=0.25cm, align = center, below = of logo](dipartimento){\normalsize \dept};
        \node[text width=40em, align = center, yshift=.55cm,below = of dipartimento](course){\normalsize \course};
        \node[text width=35em,align = center,  yshift=1.2cm,below = of course](line){\par\noindent\rule{\textwidth}{0.4pt}};
        \node[text width=40em, align = center, yshift=.55cm,below = of line](lia){\normalsize \beforetitle \xspace \subject };

        \node[text width=40em, align = center, yshift=-0.5cm,below = of lia](title){\bfseries \parbox{12cm}{\fontsize{21pt}{20pt}\selectfont \centering \title\par}};

        \node[text width=35em, align = left, yshift=-1cm,below = of title](relatoretit){\normalsize \textbf{\beforeprof} };
        \node[text width=35em, align = left, yshift=1cm,below = of relatoretit](relatore){\large \relatoreone \\ \relatoretwo};

        \node[text width=35em, align = left, yshift=0.5cm,below = of relatore](correlatoretit){\normalsize \textbf{\beforecorrelatore}  };
        \node[text width=35em, align = left, yshift=1cm,below = of correlatoretit](correlatore){\large \correlatoretwo \\ \correlatoreone};


        \node[text width=35em, align = right, yshift=-1cm,below = of title](candidatetit){\normalsize \textbf{\beforecandidate}};
        \node[text width=35em, align = right, yshift=1cm,below = of candidatetit](candidate){\large \author};

        \node[text width=35em,align = center,  yshift= -3cm,below = of candidate](line2){\par\noindent\rule{\textwidth}{0.4pt}};


        \node[text width=50em, align = center, yshift=0.5cm,below = of line2](year){\large \beforeannoacc\xspace \annoacc};
    \end{tikzpicture}
\end{titlepage}

%-------------------------------------------------------------------------------
% PARTI PRELIMINARI DEL DOCUMENTO
%-------------------------------------------------------------------------------
\cleardoublepage

%\begin{dedication}
    \begin{flushright}
        \small
        \textit{``Ad Oleg, a cui devo il 90\% degli esami''} \\
        --- Julius Robert Oppenheimer \\[2em]
        \footnotesize Dedico questo lavoro ai miei cari
    \end{flushright}
\end{dedication}
%\cleardoublepage
\pagenumbering{roman}

\tableofcontents

\cleardoublepage

%-------------------------------------------------------------------------------
% CORPO PRINCIPALE DEL DOCUMENTO
%-------------------------------------------------------------------------------
\pagenumbering{arabic}
\setcounter{page}{1}

\include{note}
\cleardoublepage

\newcommand{\myquote}[1]{``#1''}

\phantomsection%
\chapter*{Sommario}
\addcontentsline{toc}{chapter}{Sommario}

Andrà inserito il riassunto di tutta la tesi.
\chapter{Introduzione}


\section{Formato Windows PE malware}

\section{Categorie Malware}

\section{Chiamata Api Windows}

\section{Contributi}


%-------------------------------------------------------------------------------
% BIBLIOGRAFIA
%-------------------------------------------------------------------------------
\bibliographystyle{IEEEtran}
\bibliography{bibliografia}


%-------------------------------------------------------------------------------
% PARTI FINALI DEL DOCUMENTO
%-------------------------------------------------------------------------------
%\cleardoublepage
%\chapter*{Ringraziamenti}

\newpage
\null
\newpage

\newpage
\null
\newpage

\newpage
\null
\newpage

\end{document}